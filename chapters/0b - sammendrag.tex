\chapter*{Sammendrag}
Løsepengevirusangrep har sett en økning globalt de siste årene. De representerer en alvorlig trussel mot enhver bedrifts data, omdømme og operasjonelle kapabiliteter. Sikkerhetskopier er en essensiell del av en effektiv strategi for å motså løsepengevirus-angrep, med funksjonen til en siste forsvarslinje. Dermed; Hvordan skal en hindre en hacker fra å infisere eller slette sikkerhetskopier før de setter i gang med løsepengevirus-angrepet?

I denne oppgaven forsøker vi å se på de trender og karakteristiske preg som definerer moderne løsepengevirus, i tillegg til hvordan gjenopprettelse etter et moderne løsepengevirus-angrep utspiller. Vi utforsker løsninger for sikkerhetskopiering av både administrerte og ikke-administrerte databaser på Azure, Microsoft sin plattform for skytjenester, og sammenligner hvor effektive de løsningene er på å motstå moderne løsepengevirus. I tillegg ser vi på sikkerhetsmekanismer og funksjoner i Azure som bidrar til sikring av backup mot ondsinnede handlinger.
Resultatene tar for seg flere sikkerhetsfunksjoner som setter en stopper for en rekke angrepsvektorer. Videre fremhever vi problemer knyttet til eksisterende backupløsninger og i noen tilfeller foreslår vi forbedringer. 

%Skriv norsk sammendragmer
%Ransomware attacks have been on the rise globally the last few years. They pose a serious threat to any business’s data, reputation and operational capability. Back- ups are an essential part of an effective strategy against ransomware attacks, functioning as the last line of defense.But how do you prevent a hacker from corrupt- ing or outright deleting your backups before unleashing their ransomware?

%In this thesis, we attempt to find the trends and characteristics that define modern ransomware, as well as how to recover from modern ransomware attacks. We explore various backup solutions for managed and unmanaged databases in Azure, the cloud computing service operated by Microsoft, and evaluate their ef- fectiveness against modern ransomware attacks. 

%In addition, we look at security mechanisms and features in Azure that can help defend backups from malicious actions. The results address several security features which block a number of attack vectors. Furthermore, issues with the existing backup solutions are highlighted and in some cases suggestions for improvement proposed. 