\section{Encrypt ClickHouse and recover from backup} \label{app:chs1e1}
In this experiment, we install \texttt{ccrypt} (an encryption tool)
in order to encrypt database files and observe the results.
Afterwards, we recover the VM using \texttt{clickhouse-backup} and Azure Backup respectively.
We attempted to encrypt a single file, as well as an entire directory.

This experiment has two main objectives.
One is to encrypt data files used by ClickHouse and observe the results.
How will database queries be affected?
Is it possible to detect a ransomware that encrypts ClickHouse's data files?
The other objective is to recover the database after the files have been encrypted.
The goal is to learn how to configure and use both \texttt{clickhouse-backup} and Azure Backup,
as well as evaluating their usability.

\subsection{Procedure}
\label{sec:org8373240}
Files were encrypted by using the tool \texttt{ccrypt}
with an empty passphrase (example: \texttt{ccrypt -e filename}).

SQL queries were run as \texttt{azureuser} (the default, non-root user) via \texttt{clickhouse-client}.

Bash commands were also run as \texttt{azureuser}.
Commands were run in \path{/home/azureuser} unless specified otherwise.

The output of commands and queries are shown in comments (\texttt{\#} for bash and PowerShell, \texttt{-{}-} for SQL) beneath the command or query.
Comments used to explain what a command does are placed above the command.
The variables from \ref{app:ch-non-perf} need to be declared in order for many commands to work.

\subsection{Determine which files to encrypt}
\label{sec:orge561d86}
Before we could encrypt files, we had to find out which files to encrypt.

In order to find out where table data is stored,
the path to the \texttt{cell\_towers} table was fetched from \texttt{system.parts},
which, as the name suggests, contains the parts of a database table
(when using the MergeTree database engine).
A descripton of \texttt{system.parts} can be found in the ClickHouse documentation.
\cite{noauthor_parts_nodate}
% [PartsClickHouseDocs] 

Query to fetch path of the \texttt{cell\_towers} table:
\begin{minted}[breaklines=true,breakanywhere=true]{sql}
SELECT
    table,
    disk_name,
    path
FROM system.parts
WHERE table = 'cell_towers'
FORMAT TabSeparated

-- cell_towers     default /var/lib/clickhouse/store/c28/c283470d-9ab3-4be8-bd81-132274c9f9b0/all_1_35_2/
-- cell_towers     default /var/lib/clickhouse/store/c28/c283470d-9ab3-4be8-bd81-132274c9f9b0/all_36_41_1/
-- cell_towers     default /var/lib/clickhouse/store/c28/c283470d-9ab3-4be8-bd81-132274c9f9b0/all_42_42_0/
\end{minted}

We navigated to the first path listed and printed the contents of the directory (performed as root):
\begin{minted}[breaklines=true,breakanywhere=true]{bash}
cd /var/lib/clickhouse/store/c28/c283470d-9ab3-4be8-bd81-132274c9f9b0/all_1_35_2/
ls

# area.bin            cell.mrk2        count.txt                      lat.mrk2  net.bin        range.bin     unit.mrk2
# area.mrk2           changeable.bin   created.bin                    lon.bin   net.mrk2       range.mrk2    updated.bin
# averageSignal.bin   changeable.mrk2  created.mrk2                   lon.mrk2  primary.idx    samples.bin   updated.mrk2
# averageSignal.mrk2  checksums.txt    default_compression_codec.txt  mcc.bin   radio.bin.cpt  samples.mrk2
# cell.bin            columns.txt      lat.bin                        mcc.mrk2  radio.mrk2     unit.bin
\end{minted}

In order to determine which files corresponded to columns in the table,
we used \texttt{DESCRIBE TABLE} to display the columns.

\begin{minted}[breaklines=true,breakanywhere=true]{sql}
DESCRIBE TABLE cell_towers
FORMAT TabSeparated

-- radio   Enum8(\'\' = 0, \'CDMA\' = 1, \'GSM\' = 2, \'LTE\' = 3, \'NR\' = 4, \'UMTS\' = 5)
-- mcc     UInt16
-- net     UInt16
-- area    UInt16
-- cell    UInt64
-- unit    Int16
-- lon     Float64
-- lat     Float64
-- range   UInt32
-- samples UInt32
-- changeable      UInt8
-- created DateTime
-- updated DateTime
-- averageSignal   UInt8
\end{minted}

Some of the column names displayed matched the names of files in the directory.
We made an educated guess that \texttt{.bin} files are the actual data files.
A quick Google search indicates that \texttt{.mrk2} files are related to the indexing of the database.

We decided to encrypt the file \texttt{radio.bin},
since the \texttt{radio} column is used in one of the test queries we will use.

\subsection{Install \texttt{ccrypt}}
\label{sec:orga964650}
\begin{minted}[breaklines=true,breakanywhere=true]{bash}
sudo apt install ccrypt
\end{minted}
\subsection{Perform test queries}
\label{sec:org0e389d8}
In order to see how encryption would affect the database,
we first performed two test queries and noted the results.
The queries are the same as the ones used in the \ref{app:ch-non-perf}.

Test query 1:
\begin{minted}[breaklines=true,breakanywhere=true]{sql}
SELECT
    radio,
    count() AS c
FROM cell_towers
GROUP BY radio
ORDER BY c DESC
FORMAT TabSeparated

-- UMTS 20686487
-- LTE  12101148
-- GSM  9931312
-- CDMA 556344
-- NR   867
\end{minted}

Test query 2:
\begin{minted}[breaklines=true,breakanywhere=true]{sql}
SELECT
    mcc,
    count()
FROM cell_towers
GROUP BY mcc
ORDER BY count() DESC
LIMIT 10
FORMAT TabSeparated

-- 310  5024650
-- 262  2622423
-- 250  1953176
-- 208  1891187
-- 724  1836150
-- 404  1729151
-- 234  1618924
-- 510  1353998
-- 440  1343355
-- 311  1332798
\end{minted}
\subsection{Encrypt file and repeat test queries}
\label{sec:orgec4825e}
Navigate to directory and encrypt \texttt{radio.bin} (performed as root):
\begin{minted}[breaklines=true,breakanywhere=true]{bash}
cd /var/lib/clickhouse/store/c28/c283470d-9ab3-4be8-bd81-132274c9f9b0/all_1_35_2
ccrypt -e radio.bin
\end{minted}

The test queries from earlier were repeated in \texttt{clickhouse-client}.
The first test query, which selects data from the \texttt{radio} table fails,
because the file \texttt{radio.bin}, which contains a part of the table used
for the query, is encrypted.

Query using \texttt{radio}, which fails because \texttt{radio.bin} is encrypted:
\begin{minted}[breaklines=true,breakanywhere=true]{sql}
SELECT
    radio,
    count() AS c
FROM cell_towers
GROUP BY radio
ORDER BY c DESC
FORMAT TabSeparated

-- Received exception from server (version 22.4.3):
-- Code: 107. DB::Exception: Received from localhost:9000. DB::Exception: Cannot open file /var/lib/clickhouse/store/c28/c283470d-9ab3-4be8-bd81-132274c9f9b0/all_1_35_2/radio.bin, errno: 2, strerror: No such file or directory: While executing MergeTreeInOrder. (FILE_DOESNT_EXIST)
\end{minted}

The other test query, which does not use the \texttt{radio} table succeeds:
\begin{minted}[breaklines=true,breakanywhere=true]{sql}
SELECT
    mcc,
    count()
FROM cell_towers
GROUP BY mcc
ORDER BY count() DESC
LIMIT 10
FORMAT TabSeparated

-- 310  5024650
-- 262  2622423
-- 250  1953176
-- 208  1891187
-- 724  1836150
-- 404  1729151
-- 234  1618924
-- 510  1353998
-- 440  1343355
-- 311  1332798
\end{minted}
\subsection{Decrypt file and repeat test query}
\label{sec:org164be71}
The file \texttt{radio.bin.cpt} (the encrypted version of \texttt{radio.bin}) was decrypted
with \texttt{ccrypt -d radio.bin.cpt},
and the test queries were repeated.
After decrypting the file, both queries succeeded.

First test query repeated after \texttt{radio.bin.cpt} was decrypted:
\begin{minted}[breaklines=true,breakanywhere=true]{sql}
SELECT
    radio,
    count() AS c
FROM cell_towers
GROUP BY radio
ORDER BY c DESC
FORMAT TabSeparated

-- UMTS 20686487
-- LTE  12101148
-- GSM  9931312
-- CDMA 556344
-- NR   867
\end{minted}

Second test query repeated after \texttt{radio.bin.cpt} was decrypted:
\begin{minted}[breaklines=true,breakanywhere=true]{sql}
SELECT
    mcc,
    count()
FROM cell_towers
GROUP BY mcc
ORDER BY count() DESC
LIMIT 10
FORMAT TabSeparated

-- 310 5024650
-- 262 2622423
-- 250 1953176
-- 208 1891187
-- 724 1836150
-- 404 1729151
-- 234 1618924
-- 510 1353998
-- 440 1343355
-- 311 1332798
\end{minted}

\subsection{Encrypt all files in \texttt{/var/lib/clickhouse/store} and repeat test queries}
\label{sec:orgb28ee20}
In order to simulate a ransomware attack
where the attacker doesn't encrypt specific files one-by-one,
but instead encrypts the entire data directory automatically,
we encrypted the directory which stores the data parts for (non-system) ClickHouse tables.

Files in the directory \path{/var/lib/clickhouse/store},
were encrypted recursively (\texttt{-r}) using \texttt{ccrypt}.
The \texttt{-f} (force) flag makes \texttt{ccrypt} encrypt write-protected files without asking for confirmation.

Encrypting \texttt{store} (performed as root):
\begin{minted}[breaklines=true,breakanywhere=true]{bash}
ccrypt -erf store

-- Enter encryption key:
-- Enter encryption key: (repeat)
-- ccrypt: store/063/0638116a-3105-483f-a2e5-55287d4eaa27/202204_2295_2350_15: No such file or directory
-- ccrypt: store/063/0638116a-3105-483f-a2e5-55287d4eaa27/202204_2355_2355_0: No such file or directory
-- ccrypt: store/063/0638116a-3105-483f-a2e5-55287d4eaa27/202204_2351_2351_0: No such file or directory
-- ccrypt: store/11f/11fafbd1-2c3f-4621-a574-b4f7b5594988/202204_2794_2794_0: No such file or directory
-- ccrypt: store/11f/11fafbd1-2c3f-4621-a574-b4f7b5594988/202204_2140_2793_477: No such file or directory
\end{minted}

The contents of one of the encryted directories (performed as root):
\begin{minted}[breaklines=true,breakanywhere=true]{bash}
ls -l /var/lib/clickhouse/store/c28/c283470d-9ab3-4be8-bd81-132274c9f9b0/all_1_35_2

# Result:
# total 918640
# -rw-r----- 1 clickhouse clickhouse  41208741 Apr 26 13:09 area.bin.cpt
# -rw-r----- 1 clickhouse clickhouse    107576 Apr 26 13:09 area.mrk2.cpt
# -rw-r----- 1 clickhouse clickhouse    163548 Apr 26 13:09 averageSignal.bin.cpt
# -rw-r----- 1 clickhouse clickhouse    107576 Apr 26 13:09 averageSignal.mrk2.cpt
# -rw-r----- 1 clickhouse clickhouse 157679630 Apr 26 13:09 cell.bin.cpt
# -rw-r----- 1 clickhouse clickhouse    107576 Apr 26 13:09 cell.mrk2.cpt
# -rw-r----- 1 clickhouse clickhouse    163548 Apr 26 13:09 changeable.bin.cpt
# -rw-r----- 1 clickhouse clickhouse    107576 Apr 26 13:09 changeable.mrk2.cpt
# -rw-r----- 1 clickhouse clickhouse      1183 Apr 26 13:09 checksums.txt.cpt
# -rw-r----- 1 clickhouse clickhouse       354 Apr 26 13:09 columns.txt.cpt
# -rw-r----- 1 clickhouse clickhouse        40 Apr 26 13:09 count.txt.cpt
# -rw-r----- 1 clickhouse clickhouse  65728056 Apr 26 13:09 created.bin.cpt
# -rw-r----- 1 clickhouse clickhouse    107576 Apr 26 13:09 created.mrk2.cpt
# -rw-r----- 1 clickhouse clickhouse        42 Apr 26 13:09 default_compression_codec.txt.cpt
# -rw-r----- 1 clickhouse clickhouse 224774675 Apr 26 13:09 lat.bin.cpt
# -rw-r----- 1 clickhouse clickhouse    107576 Apr 26 13:09 lat.mrk2.cpt
# -rw-r----- 1 clickhouse clickhouse 227815661 Apr 26 13:09 lon.bin.cpt
# -rw-r----- 1 clickhouse clickhouse    107576 Apr 26 13:09 lon.mrk2.cpt
# -rw-r----- 1 clickhouse clickhouse    331043 Apr 26 13:09 mcc.bin.cpt
# -rw-r----- 1 clickhouse clickhouse    107576 Apr 26 13:09 mcc.mrk2.cpt
# -rw-r----- 1 clickhouse clickhouse    345344 Apr 26 13:09 net.bin.cpt
# -rw-r----- 1 clickhouse clickhouse    107576 Apr 26 13:09 net.mrk2.cpt
# -rw-r----- 1 clickhouse clickhouse     40361 Apr 26 13:09 primary.idx.cpt
# -rw-r----- 1 clickhouse clickhouse    163563 Apr 26 13:09 radio.bin.cpt
# -rw-r----- 1 clickhouse clickhouse    107576 Apr 26 13:09 radio.mrk2.cpt
# -rw-r----- 1 clickhouse clickhouse  41566798 Apr 26 13:09 range.bin.cpt
# -rw-r----- 1 clickhouse clickhouse    107576 Apr 26 13:09 range.mrk2.cpt
# -rw-r----- 1 clickhouse clickhouse  66713118 Apr 26 13:09 samples.bin.cpt
# -rw-r----- 1 clickhouse clickhouse    107576 Apr 26 13:09 samples.mrk2.cpt
# -rw-r----- 1 clickhouse clickhouse   1331731 Apr 26 13:09 unit.bin.cpt
# -rw-r----- 1 clickhouse clickhouse    107576 Apr 26 13:09 unit.mrk2.cpt
# -rw-r----- 1 clickhouse clickhouse 110977534 Apr 26 13:09 updated.bin.cpt
# -rw-r----- 1 clickhouse clickhouse    107576 Apr 26 13:09 updated.mrk2.cpt
\end{minted}

Then the test queries were repeated.

Test query 1 fails:
\begin{minted}[breaklines=true,breakanywhere=true]{sql}
SELECT
    mcc,
    count()
FROM cell_towers
GROUP BY mcc
ORDER BY count() DESC
LIMIT 10
FORMAT TabSeparated

-- Query id: 6d32968b-85bb-4059-9119-ae76a84b8c13
--
-- 0 rows in set. Elapsed: 0.023 sec.
--
-- Received exception from server (version 22.4.3):
-- Code: 107. DB::Exception: Received from localhost:9000. DB::Exception: Cannot open file /var/lib/clickhouse/store/c28/c283470d-9ab3-4be8-bd81-132274c9f9b0/all_1_35_2/mcc.bin, errno: 2, strerror: No such file or directory: While executing MergeTreeInOrder. (FILE_DOESNT_EXIST)
\end{minted}

Test query 2 also fails:
\begin{minted}[breaklines=true,breakanywhere=true]{sql}
SELECT
    radio,
    count() AS c
FROM cell_towers
GROUP BY radio
ORDER BY c DESC
FORMAT TabSeparated

-- Query id: 8c913fd5-b0db-4cc2-b176-f389886c3bce
--
-- 0 rows in set. Elapsed: 0.003 sec.
--
-- Received exception from server (version 22.4.3):
-- Code: 107. DB::Exception: Received from localhost:9000. DB::Exception: Cannot open file /var/lib/clickhouse/store/c28/c283470d-9ab3-4be8-bd81-132274c9f9b0/all_1_35_2/radio.bin, errno: 2, strerror: No such file or directory: While executing MergeTreeInOrder. (FILE_DOESNT_EXIST)
\end{minted}

Selecting all columns fails:
\begin{minted}[breaklines=true,breakanywhere=true]{sql}
SELECT *
FROM cell_towers
LIMIT 10

-- Query id: 7a96d377-2f5b-4439-96d6-eb554b21fea7
--
-- 0 rows in set. Elapsed: 0.003 sec.
--
-- Received exception from server (version 22.4.3):
-- Code: 107. DB::Exception: Received from localhost:9000. DB::Exception: Cannot open file /var/lib/clickhouse/store/c28/c283470d-9ab3-4be8-bd81-132274c9f9b0/all_1_35_2/radio.bin, errno: 2, strerror: No such file or directory: While executing MergeTreeInOrder. (FILE_DOESNT_EXIST)
\end{minted}

Selecting any single column seems to fail:
\begin{minted}[breaklines=true,breakanywhere=true]{sql}
SELECT updated
FROM cell_towers
LIMIT 10

-- Query id: c983d27d-144d-4296-adb3-ade5ebba3777
--
--
-- 0 rows in set. Elapsed: 0.002 sec.
--
-- Received exception from server (version 22.4.3):
-- Code: 107. DB::Exception: Received from localhost:9000. DB::Exception: Cannot open file /var/lib/clickhouse/store/c28/c283470d-9ab3-4be8-bd81-132274c9f9b0/all_1_35_2/updated.bin, errno: 2, strerror: No such file or directory: While executing MergeTreeInOrder. (FILE_DOESNT_EXIST)
\end{minted}

\subsection{Rebuild VM and try to recover}
\label{sec:orgeb16138}
\subsubsection{Delete VM}
\label{sec:orga4a9443}
Script to delete the VM and accociated resources (run in Azure Cloud Shell):
\begin{minted}[breaklines=true,breakanywhere=true]{powershell}
# Delete the VM
az vm delete --name $CHName --resource-group $RGName --yes

# Get all resources in resource group
$resources = az resource list --resource-group $RGName | ConvertFrom-Json -AsHashtable

# Fetch only the ids of resources with names containing "clickhouse".
# The VM was named "clickhouseVM", which is why this works.
# If the VM had a different name, we would have to grep for something else.
$filtered = foreach($r in $resources) {
    Write-Output $r["id"] | grep clickhouse
}

# Delete the resources
$filtered | % {Remove-AzResource -ResourceId $_ -Force}
$filtered | % {Remove-AzResource -ResourceId $_ -Force}
\end{minted}

The last command had to be run twice,
because some resources could not be deleted while other resources depended on them.
Repeating the command had the effect of deleting ``child'' resources first,
so that ``parent'' resources could be deleted.

\subsubsection{Set up VM according to test environment setup}
\label{sec:orge14ae6f}
Outputs were skipped for the sake of brevity.
Everything went as planned.

Create VM (run from cloud shell):
\begin{minted}[breaklines=true,breakanywhere=true]{powershell}
az vm create `
    --resource-group $RGName `
    --name $CHName `
    --image Canonical:UbuntuServer:16.04-LTS:16.04.202109280 `
    --admin-username azureuser `
    --size Standard_B1s `
    --ssh-key-values $SSHPath `
    --public-ip-sku Standard
\end{minted}

Install clickhouse (run from bash in VM):
\begin{minted}[breaklines=true,breakanywhere=true]{bash}
sudo apt-get install -y apt-transport-https ca-certificates dirmngr
sudo apt-key adv --keyserver hkp://keyserver.ubuntu.com:80 --recv 8919F6BD2B48D754

echo "deb https://packages.clickhouse.com/deb stable main" | sudo tee \
    /etc/apt/sources.list.d/clickhouse.list
sudo apt-get update

sudo apt-get install -y clickhouse-server clickhouse-client

sudo service clickhouse-server start
\end{minted}

Install clickhouse-backup (run from bash in VM):
\begin{minted}[breaklines=true,breakanywhere=true]{bash}
# Download archive containing binary
wget https://github.com/AlexAkulov/clickhouse-backup/releases/download/v1.3.2/clickhouse-backup-linux-amd64.tar.gz

# Decompress archive
tar -zxvf clickhouse-backup-linux-amd64.tar.gz

# Move binary to home directory
mv build/linux/amd64/clickhouse-backup ~

# Cleanup
rmdir -p build/linux/amd64
rm clickhouse-backup-linux-amd64.tar.gz
\end{minted}

The \texttt{clickhouse-backup} config file was copied from
the test environment setup (see \ref{app:ch-non-perf})
and saved as \path{~/config.yaml}.

\subsubsection{Restore from remote backup}
\label{sec:org8a05b22}
Commands were run from bash on the VM.

List remote backups:
\begin{minted}[breaklines=true,breakanywhere=true]{bash}
sudo ./clickhouse-backup list remote --config config.yaml

# 2022/05/03 10:38:55.223041  info SELECT max(toInt64(bytes_on_disk * 1.02)) AS max_file_size FROM system.parts
# 2022-05-02T09-48-18   1.07GiB   02/05/2022 09:48:39   remote      tar
\end{minted}

Restore from the remote backup:
\begin{minted}[breaklines=true,breakanywhere=true]{bash}
sudo ./clickhouse-backup restore_remote 2022-05-02T09-48-18 --config config.yaml

# 2022/05/03 10:40:28.403253  info SELECT value FROM `system`.`build_options` where name='VERSION_INTEGER'
# 2022/05/03 10:40:28.407883  info SELECT * FROM system.disks;
# 2022/05/03 10:40:28.422520  info SELECT max(toInt64(bytes_on_disk * 1.02)) AS max_file_size FROM system.parts
# 2022/05/03 10:40:28.512501  info done                      backup=2022-05-02T09-48-18 duration=10ms operation=download size=765B table_metadata=default.cell_towers
# 2022/05/03 10:40:48.487522  info done                      diff_parts=0 duration=0s operation=downloadDiffParts===================================================================================================================] 100.00% 15s
# 2022/05/03 10:40:49.252289  info done                      backup=2022-05-02T09-48-18 duration=20.739s operation=download_data size=1.07GiB table=default.cell_towers
# 2022/05/03 10:40:49.269557  info done                      backup=2022-05-02T09-48-18 duration=20.855s operation=download size=1.07GiB
# 2022/05/03 10:40:49.326168  info SELECT value FROM `system`.`build_options` where name='VERSION_INTEGER'
# 2022/05/03 10:40:49.420143  info SELECT * FROM system.disks;
# 2022/05/03 10:40:49.429084  info CREATE DATABASE IF NOT EXISTS default
# ENGINE = Atomic
# 2022/05/03 10:40:49.437031  info SELECT engine FROM system.databases WHERE name = 'default'
# 2022/05/03 10:40:49.440420  info DROP TABLE IF EXISTS `default`.`cell_towers` NO DELAY
# 2022/05/03 10:40:49.444862  info CREATE DATABASE IF NOT EXISTS `default`
# 2022/05/03 10:40:49.448408  info CREATE TABLE default.cell_towers UUID 'f9b9b44f-3af6-4bf5-9458-9ef6a81a3519' (`radio` Enum8('' = 0, 'CDMA' = 1, 'GSM' = 2, 'LTE' = 3, 'NR' = 4, 'UMTS' = 5), `mcc` UInt16, `net` UInt16, `area` UInt16, `cell` UInt64, `unit` Int16, `lon` Float64, `lat` Float64, `range` UInt32, `samples` UInt32, `changeable` UInt8, `created` DateTime, `updated` DateTime, `averageSignal` UInt8) ENGINE = MergeTree ORDER BY (radio, mcc, net, created) SETTINGS index_granularity = 8192
# 2022/05/03 10:40:49.498743  info SELECT count() FROM system.settings WHERE name = 'show_table_uuid_in_table_create_query_if_not_nil'
# 2022/05/03 10:40:49.519183  info SELECT name FROM system.databases WHERE engine IN ('MySQL','PostgreSQL')
# 2022/05/03 10:40:49.533985  info
#                 SELECT
#                         countIf(name='data_path') is_data_path_present,
#                         countIf(name='data_paths') is_data_paths_present,
#                         countIf(name='uuid') is_uuid_present,
#                         countIf(name='create_table_query') is_create_table_query_present,
#                         countIf(name='total_bytes') is_total_bytes_present
#                 FROM system.columns WHERE database='system' AND table='tables'
#
# 2022/05/03 10:40:49.545657  info SELECT database, name, engine , data_paths , uuid , create_table_query , coalesce(total_bytes, 0) AS total_bytes   FROM system.tables WHERE is_temporary = 0 SETTINGS show_table_uuid_in_table_create_query_if_not_nil=1
# 2022/05/03 10:40:49.566479  info SELECT sum(bytes_on_disk) as size FROM system.parts WHERE database='default' AND table='cell_towers' GROUP BY database, table
# 2022/05/03 10:40:49.578319  info ALTER TABLE `default`.`cell_towers` ATTACH PART 'all_1_35_2'
# 2022/05/03 10:40:49.589570  info ALTER TABLE `default`.`cell_towers` ATTACH PART 'all_36_41_1'
# 2022/05/03 10:40:49.596540  info ALTER TABLE `default`.`cell_towers` ATTACH PART 'all_42_42_0'
# 2022/05/03 10:40:49.601181  info done                      backup=2022-05-02T09-48-18 operation=restore table=default.cell_towers
# 2022/05/03 10:40:49.601380  info done                      backup=2022-05-02T09-48-18 duration=109ms operation=restore
# 2022/05/03 10:40:49.601555  info done                      backup=2022-05-02T09-48-18 operation=restore
\end{minted}

\subsubsection{Repeat test queries}
\label{sec:org4f2f2e4}
Test queries were run from \texttt{clickhouse-backup}
and compared with earlier results to se if they were valid.
Both queries returned valid results, indicating that the recovery was successful.

At first, performance was very poor.
Sometimes the queries would time out because they took too long (more than 300 seconds).
After restarting the VM, queries were as fast as usual (0.072s and 0.2s respectively).

Test query 1:
\begin{minted}[breaklines=true,breakanywhere=true]{sql}
SELECT
    radio,
    count() AS c
FROM cell_towers
GROUP BY radio
ORDER BY c DESC
FORMAT TabSeparated

-- Query id: fea063fa-98e6-4a99-b373-4f51d99be3c1
--
-- UMTS    20686487
-- LTE     12101148
-- GSM     9931312
-- CDMA    556344
-- NR      867
\end{minted}

Test query 2:
\begin{minted}[breaklines=true,breakanywhere=true]{sql}
SELECT
    mcc,
    count()
FROM cell_towers
GROUP BY mcc
ORDER BY count() DESC
LIMIT 10
FORMAT TabSeparated

-- Query id: e4cd8484-7d4e-4e97-b78f-5373ed61c2b2
--
-- 310     5024650
-- 262     2622423
-- 250     1953176
-- 208     1891187
-- 724     1836150
-- 404     1729151
-- 234     1618924
-- 510     1353998
-- 440     1343355
-- 311     1332798
\end{minted}

\subsection{Repeat encryption and recover database with Azure Backup}
\label{sec:org2f39a38}
\subsubsection{Encrypt \texttt{store} directory}
\label{sec:org53cf4ea}
Run as root in bash on the VM:
\begin{minted}[breaklines=true,breakanywhere=true]{bash}
ccrypt -erf /var/lib/clickhouse/store
\end{minted}
\subsubsection{Delete VM}
\label{sec:org3c61e5e}
Deleting the VM:
\begin{minted}[breaklines=true,breakanywhere=true]{powershell}
az vm delete --name $CHName --resource-group $RGName --yes
\end{minted}

When using Azure Backup via the Azure Portal to recover a VM,
a virtual network and a subnet have to be provided.
We assume the same holds true for recovering via the Azure CLI,
even though no commands seem to be asking for the aforementioned resources (at least not directly).
This is why we only deleted the VM, and not related resources like network cards and IP adresses as well.

\subsubsection{Create a Storage Account}
\label{sec:orgab49f59}
In order to recover the VM, a storage account is required for staging the recovery.

Creating a storage account:
\begin{minted}[breaklines=true,breakanywhere=true]{powershell}
az storage account create `
    --name $StagingSAName `
    --resource-group $RGName `
    --location eastus `
    --sku Standard_LRS
\end{minted}

Output:
\begin{minted}[breaklines=true,breakanywhere=true]{json}
{
  "accessTier": "Hot",
  "allowBlobPublicAccess": true,
  "allowCrossTenantReplication": null,
  "allowSharedKeyAccess": null,
  "allowedCopyScope": null,
  "azureFilesIdentityBasedAuthentication": null,
  "blobRestoreStatus": null,
  "creationTime": "2022-05-04T07:11:01.403417+00:00",
  "customDomain": null,
  "defaultToOAuthAuthentication": null,
  "dnsEndpointType": null,
  "enableHttpsTrafficOnly": true,
  "enableNfsV3": null,
  "encryption": {
    "encryptionIdentity": null,
    "keySource": "Microsoft.Storage",
    "keyVaultProperties": null,
    "requireInfrastructureEncryption": null,
    "services": {
      "blob": {
        "enabled": true,
        "keyType": "Account",
        "lastEnabledTime": "2022-05-04T07:11:01.528489+00:00"
      },
      "file": {
        "enabled": true,
        "keyType": "Account",
        "lastEnabledTime": "2022-05-04T07:11:01.528489+00:00"
      },
      "queue": null,
      "table": null
    }
  },
  "extendedLocation": null,
  "failoverInProgress": null,
  "geoReplicationStats": null,
  "id": "/subscriptions/4b48eb85-91f3-4902-b74b-e84641fb6785/resourceGroups/testRG/providers/Microsoft.Storage/storageAccounts/stagingchsa",
  "identity": null,
  "immutableStorageWithVersioning": null,
  "isHnsEnabled": null,
  "isLocalUserEnabled": null,
  "isSftpEnabled": null,
  "keyCreationTime": {
    "key1": "2022-05-04T07:11:01.528489+00:00",
    "key2": "2022-05-04T07:11:01.528489+00:00"
  },
  "keyPolicy": null,
  "kind": "StorageV2",
  "largeFileSharesState": null,
  "lastGeoFailoverTime": null,
  "location": "eastus",
  "minimumTlsVersion": "TLS1_0",
  "name": "stagingchsa",
  "networkRuleSet": {
    "bypass": "AzureServices",
    "defaultAction": "Allow",
    "ipRules": [],
    "resourceAccessRules": null,
    "virtualNetworkRules": []
  },
  "primaryEndpoints": {
    "blob": "https://stagingchsa.blob.core.windows.net/",
    "dfs": "https://stagingchsa.dfs.core.windows.net/",
    "file": "https://stagingchsa.file.core.windows.net/",
    "internetEndpoints": null,
    "microsoftEndpoints": null,
    "queue": "https://stagingchsa.queue.core.windows.net/",
    "table": "https://stagingchsa.table.core.windows.net/",
    "web": "https://stagingchsa.z13.web.core.windows.net/"
  },
  "primaryLocation": "eastus",
  "privateEndpointConnections": [],
  "provisioningState": "Succeeded",
  "publicNetworkAccess": null,
  "resourceGroup": "testRG",
  "routingPreference": null,
  "sasPolicy": null,
  "secondaryEndpoints": null,
  "secondaryLocation": null,
  "sku": {
    "name": "Standard_LRS",
    "tier": "Standard"
  },
  "statusOfPrimary": "available",
  "statusOfSecondary": null,
  "storageAccountSkuConversionStatus": null,
  "tags": {},
  "type": "Microsoft.Storage/storageAccounts"
}
\end{minted}

\subsubsection{Recover from Azure Backup}
\label{sec:orgd58c2f9}
The VM was recovered by modifying instructions in the documentation \cite{v-amallick_back_nodate-1}
to suit our use case.
% [v-amallickBackRecoverAzure]
Many steps, like creating a Recovery Services vault, had already been performed and were thus skipped.
The commands we performed to recover the VM are listed below.
The process consists of retrieving information used to list relevant recovery points.
Then, a restore job is started in order to recover the VHD.
When the restore job is finished,
it produces a template which is then used to deploy a new VM.

Initial setup:
\begin{minted}[breaklines=true,breakanywhere=true]{powershell}
# Get RSV and set context
$RSV = Get-AzRecoveryServicesVault -Name $RSVName -ResourceGroupName $RGName
Set-AzRecoveryServicesVaultContext -Vault $RSV

# Select VM
$namedContainer = Get-AzRecoveryServicesBackupContainer  -ContainerType "AzureVM" -Status "Registered" -FriendlyName $CHName -VaultId $RSV.ID
$backupitem = Get-AzRecoveryServicesBackupItem -Container $namedContainer  -WorkloadType "AzureVM" -VaultId $RSV.ID

# Get start and end date
$startDate = (Get-Date).AddDays(-7)
$endDate = Get-Date

# Store recovery points in variable
$rp = Get-AzRecoveryServicesBackupRecoveryPoint -Item $backupitem -StartDate $startdate.ToUniversalTime() -EndDate $enddate.ToUniversalTime() -VaultId $RSV.ID
\end{minted}

Show contents of \texttt{rp}:
\begin{minted}[breaklines=true,breakanywhere=true]{powershell}
$rp

# RecoveryPointId    RecoveryPointType  RecoveryPointTime      ContainerName                        ContainerType
# ---------------    -----------------  -----------------      -------------                        -------------
# 201217219254041    CrashConsistent    5/4/2022 9:30:41 AM    iaasvmcontainerv2;testrg;clickhouse… AzureVM
# 208189198246820    CrashConsistent    5/2/2022 10:35:59 PM   iaasvmcontainerv2;testrg;clickhouse… AzureVM
# 210069911701143    CrashConsistent    5/2/2022 12:37:15 PM   iaasvmcontainerv2;testrg;clickhouse… AzureVM
\end{minted}

The recovery point \texttt{\$rp[1]} (208189198246820) is the one that was triggered manually earlier (see \ref{app:ch-non-perf}).
It was chosen to be used for recovery.

A restore job was started:
\begin{minted}[breaklines=true,breakanywhere=true]{powershell}
# Select recovery point
$RecPoint = $rp[1]

# Create a restore job for the backup item
$restorejob = Restore-AzRecoveryServicesBackupItem -RecoveryPoint $RecPoint -StorageAccountName $StagingSAName -StorageAccountResourceGroupName $RGName -TargetResourceGroupName $RGName -VaultId $RSV.ID

# Wait for the restore job to complete
Wait-AzRecoveryServicesBackupJob -Job $restorejob -Timeout 43200

# Get details of the restore job
$restorejob = Get-AzRecoveryServicesBackupJob -Job $restorejob -VaultId $RSV.ID
$details = Get-AzRecoveryServicesBackupJobDetail -Job $restorejob -VaultId $RSV.ID

# Print details
$details
# WorkloadName     Operation            Status               StartTime                 EndTime                   JobID
# ------------     ---------            ------               ---------                 -------                   -----
# clickhousevm     Restore              Completed            5/4/2022 9:30:38 AM       5/4/2022 9:51:17 AM       3019de90-58c6-4f40-9157-cd53c7546223
\end{minted}

We then declared some variables based on the details, to be used for future commands:
\begin{minted}[breaklines=true,breakanywhere=true]{powershell}
$properties = $details.properties
$storageAccountName = $properties["Target Storage Account Name"]
$containerName = $properties["Config Blob Container Name"]
$templateBlobURI = $properties["Template Blob Uri"]
\end{minted}

The contents of \texttt{\$properties}:
\begin{minted}[breaklines=true,breakanywhere=true]{text}
Key                         Value
---                         -----
Job Type                    Recover disks
Target VM Name              vmName
Target Storage Account Name stagingchsa
Recovery point time         5/2/2022 10:35:59 PM
Config Blob Name            config-clickhousevm-19ee2306-41c2-49bf-896f-40aa8172d2ea.json
Config Blob Container Name  clickhousevm-0edd1295169343bda63beb53c8781b85
Config Blob Uri             https://stagingchsa.blob.core.windows.net/clickhousevm-0edd1295169343bda63beb53c8781b85/config-clickhousevm-19ee2306-41c2-49bf-896f-40aa8172d2ea.j…
Target resource group       testRG
Template Blob Uri           https://stagingchsa.blob.core.windows.net/clickhousevm-0edd1295169343bda63beb53c8781b85/azuredeploy19ee2306-41c2-49bf-896f-40aa8172d2ea.json
\end{minted}

Deploying the VM from the template:
\begin{minted}[breaklines=true,breakanywhere=true]{powershell}
# Template name was copied from the "Template Blob Uri"
$templateName = "azuredeploy19ee2306-41c2-49bf-896f-40aa8172d2ea.json"

# Set the storage account
Set-AzCurrentStorageAccount -Name $storageAccountName -ResourceGroupName $RGName

# Generate SAS token
$templateBlobFullURI = New-AzStorageBlobSASToken -Container $containerName -Blob $templateName -Permission r -FullUri

# Deploy VM (VirtualMachineName had to be specified interactively)
New-AzResourceGroupDeployment -Name $CHName -ResourceGroupName $RGName -TemplateUri $templateBlobFullURI
\end{minted}

Output from the \texttt{New-AzResourceGroupDeployment} (The VirtualMachineName had to be supplied via user input, indentation was adjusted for readability):
\begin{minted}[breaklines=true,breakanywhere=true]{text}
cmdlet New-AzResourceGroupDeployment at command pipeline position 1
Supply values for the following parameters:
(Type !? for Help.)
VirtualMachineName: clickhouseVM

DeploymentName          : clickhouseVM
ResourceGroupName       : testRG
ProvisioningState       : Succeeded
Timestamp               : 5/4/2022 11:50:42 AM
Mode                    : Incremental
TemplateLink            :
Uri            : https://stagingchsa.blob.core.windows.net/clickhousevm-0edd1295169343bda63beb53c8781b85/azuredeploy19ee2306-41c2-49bf-896f-40aa8172d
2ea.json?sv=2021-04-10&se=2022-05-04T12%3A46%3A51Z&sr=b&sp=r&sig=Le%2BGX1JoU%2FSivJi9EpXyk0bPHAXxBsww%2BxY7SpGsP9k%3D
ContentVersion : 1.0.0.0
Parameters              :
Name                           Type                       Value
=============================  =========================  ==========
virtualMachineName             String                     "clickhouseVM"
virtualNetwork                 String                     "clickhouseVMVNET"
virtualNetworkResourceGroup    String                     "testRG"
subnet                         String                     "clickhouseVMSubnet"
osDiskName                     String                     "clickhouseVMOSDisk"
networkInterfacePrefixName     String                     "clickhouseVMRestoredNIC"
publicIpAddressName            String                     "clickhouseVMRestoredip"

Outputs                 :
DeploymentDebugLogLevel :
\end{minted}
\subsubsection{Repeat test queries}
\label{sec:org704b2f8}
The test queries were repeated in \texttt{clickhouse-client}.
Both were successful and reasonably fast.

Test query 1:
\begin{minted}[breaklines=true,breakanywhere=true]{sql}
SELECT
    radio,
    count() AS c
FROM cell_towers
GROUP BY radio
ORDER BY c DESC
FORMAT TabSeparated

-- Query id: fa3847f4-e564-4c8b-b2be-e85e2d4db87e
--
-- UMTS	20686487
-- LTE	12101148
-- GSM	9931312
-- CDMA	556344
-- NR	867
--
-- 5 rows in set. Elapsed: 0.094 sec. Processed 43.28 million rows, 43.28 MB (462.68 million rows/s., 462.68 MB/s.)
\end{minted}

Test query 2:
\begin{minted}[breaklines=true,breakanywhere=true]{sql}
SELECT
    mcc,
    count()
FROM cell_towers
GROUP BY mcc
ORDER BY count() DESC
LIMIT 10
FORMAT TabSeparated

-- Query id: f944c5ad-937c-45fa-a9f1-d173d12f7dbc
--
-- 310	5024650
-- 262	2622423
-- 250	1953176
-- 208	1891187
-- 724	1836150
-- 404	1729151
-- 234	1618924
-- 510	1353998
-- 440	1343355
-- 311	1332798
--
-- 10 rows in set. Elapsed: 0.392 sec. Processed 43.28 million rows, 86.55 MB (110.54 million rows/s., 221.07 MB/s.)
\end{minted}
